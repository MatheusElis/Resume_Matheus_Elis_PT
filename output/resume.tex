% Copyright 2013 Christophe-Marie Duquesne <chmd@chmd.fr>
% Copyright 2014 Mark Szepieniec <http://github.com/mszep>
% 
% ConText style for making a resume with pandoc. Inspired by moderncv.
% 
% This CSS document is delivered to you under the CC BY-SA 3.0 License.
% https://creativecommons.org/licenses/by-sa/3.0/deed.en_US

\startmode[*mkii]
  \enableregime[utf-8]  
  \setupcolors[state=start]
\stopmode

\setupcolor[hex]
\definecolor[titlegrey][h=757575]
\definecolor[sectioncolor][h=394b72]
\definecolor[rulecolor][h=7570b7]

% Enable hyperlinks
\setupinteraction[state=start, color=sectioncolor]

\setuppapersize [A4][A4]
\setuplayout    [width=middle, height=middle,
                 backspace=20mm, cutspace=0mm,
                 topspace=10mm, bottomspace=20mm,
                 header=0mm, footer=0mm]

%\setuppagenumbering[location={footer,center}]

\setupbodyfont[11pt, helvetica]

\setupwhitespace[medium]

\setupblackrules[width=31mm, color=rulecolor]

\setuphead[chapter]      [style=\tfd]
\setuphead[section]      [style=\tfd\bf, color=titlegrey, align=middle]
\setuphead[subsection]   [style=\tfb\bf, color=sectioncolor, align=right,
                          before={\leavevmode\blackrule\hspace}]
\setuphead[subsubsection][style=\bf]

\setuphead[chapter, section, subsection, subsubsection][number=no]

%\setupdescriptions[width=10mm]

\definedescription
  [description]
  [headstyle=bold, style=normal,
   location=hanging, width=18mm, distance=14mm, margin=0cm]

\setupitemize[autointro, packed]    % prevent orphan list intro
\setupitemize[indentnext=no]

\defineitemgroup[enumerate]
\setupenumerate[each][fit][itemalign=left,distance=.5em,style={\feature[+][default:tnum]}]

\setupfloat[figure][default={here,nonumber}]
\setupfloat[table][default={here,nonumber}]

\setuptables[textwidth=max, HL=none]
\setupxtable[frame=off,option={stretch,width}]

\setupthinrules[width=15em] % width of horizontal rules

\setupdelimitedtext
  [blockquote]
  [before={\setupalign[middle]},
   indentnext=no,
  ]


\starttext

\section[title={Matheus Elis da
Silva},reference={matheus-elis-da-silva}]

\thinrule

\startblockquote
Bacharel em Física com ênfase em Física Computacional pela Universidade
Federal Fluminense. Engenheiro de Software com experiência em
desenvolvimento de software, criação e manutenção de processos ETL,
automação de processos de dados, arquitetura de software em nuvem e
microserviços.
\stopblockquote

\thinrule

\subsection[title={Formação Acadêmica},reference={formação-acadêmica}]

\startdescription{2015-2021}
  {\bf Bacharel, Física com ênfase em Física Computacional};
  Universidade Federal Fluminense

  {\em Monitor Bolsista no Projeto Sensebiliza (2017-2021)}

  {\em Monografia intitulada: Estudo da Dinâmica da Hanseníase no Norte
  do Brasil Via Modelagem Baseada em Agentes}
\stopdescription

\subsection[title={Experiência
Profissional},reference={experiência-profissional}]

{\bf Desenvolvedor de Software FullStack Pleno: 2RP Net} - {\em out de
2022 - Atual}

\startitemize[packed]
\item
  Desenvolvimento e manutenção de aplicações Backend utilizando as
  liguagens Python e Golang.
\item
  Criação e manutenção de pipelines de ETL para o setor de integração de
  dados.
\item
  Manutenção e gerenciamento de bancos de dados SQL (PostgreSQL, MySQL e
  OracleDB) e NoSQL (MongoDB, Google Datastore e Redis).
\item
  Criação de Microserviços em ambientes de Nuvem (Google Cloud Run,
  Google Cloud Function e Google Kubernetes Engine).
\item
  Monitoramento de aplicações utilizando Grafana, Google Cloud Logging e
  Google Cloud Monitoring.
\stopitemize

{\bf Desenvolvedor de Software Back-End Pleno: ZAX} - {\em mar de 2022 -
set de 2022}

\startitemize[packed]
\item
  Desenvolvimento e manutenção de aplicações Back-End utilizando as
  linguagens Python e NodeJS.
\item
  Desenvolvimento do ERP interno da empresa.
\item
  Automatização do CI/CD utilizando GitHub Actions e Serverless
  Framework.
\item
  Utilização de ferramentas de Nuvem como AWS Lambda, AWS ECS, AWS Cloud
  Watch e AWS SQS.
\item
  Manutenção e gerenciamento de bancos de dados SQL (PostgreSQL e
  MySQL).
\stopitemize

{\bf Desenvolvedor de Software Back-End Pleno: TrackCash} - {\em jun de
2021 - abr de 2022}

\startitemize[packed]
\item
  Desenvolvimento e manutenção de aplicações de automação de dados
  utilizando Python.
\item
  Desenvolvimento e manutenção de aplicações para monitorar rotinas de
  automação.
\item
  Desenvolvimento de aplicações internas de CLI para gerenciar rotinas
  de ETL.
\stopitemize

\subsection[title={Habilidades
Técnicas},reference={habilidades-técnicas}]

{\bf Linguagens de Programação}

\startitemize[packed]
\item
  Python
\item
  JavaScript
\item
  TypeScript
\item
  Golang
\item
  HTML
\item
  CSS
\item
  Lua
\item
  PHP
\stopitemize

{\bf Frameworks e ferramentas}

\startitemize[packed]
\item
  FastAPI
\item
  Flask
\item
  Django
\item
  Spark
\item
  Express
\item
  React
\item
  Bootstrap
\item
  Laravel
\item
  Gin
\item
  Docker
\item
  Docker Compose
\item
  Kubernets
\item
  Terraform
\stopitemize

\subsection[title={Soft Skill},reference={soft-skill}]

\startitemize
\item
  Idiomas:

  \startitemize[packed]
  \item
    Português (Nativo),
  \item
    Ingles (Intermediário),
  \item
    Espanhol (Básico).
  \stopitemize
\item
  Metodologias Ágeis:

  \startitemize[packed]
  \item
    Scrum,
  \item
    Kanban.
  \stopitemize
\stopitemize

\thinrule

\startblockquote
\useURL[url1][mailto:matheus.elis.silva@gmail.com][][matheus.elis.silva@gmail.com]\from[url1]
• +55 (12) 99669 4139 • 27 anos\crlf
\useURL[url2][https://www.linkedin.com/in/matheuselisdasilva]\from[url2]\crlf
\useURL[url3][https://github.com/MatheusElis]\from[url3]\crlf
Barueri - São Paulo, Brasil
\stopblockquote

\stoptext
